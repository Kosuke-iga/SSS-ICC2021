\documentclass[conference]{IEEEtran}
\IEEEoverridecommandlockouts
% The preceding line is only needed to identify funding in the first footnote. If that is unneeded, please comment it out.
\usepackage{cite}
\usepackage{amsmath,amssymb,amsfonts}
\usepackage{algorithmic}
\usepackage{graphicx}
\usepackage{textcomp}
\usepackage{xcolor}
\def\BibTeX{{\rm B\kern-.05em{\sc i\kern-.025em b}\kern-.08em
    T\kern-.1667em\lower.7ex\hbox{E}\kern-.125emX}}
\begin{document}

\title{Conference Paper Title*\\
{\footnotesize \textsuperscript{*}Note: Sub-titles are not captured in Xplore and
should not be used}
\thanks{identify applicable funding agency here. If none, delete this.}
}

\author{\IEEEauthorblockN{1\textsuperscript{st} Given Name Surname}
\IEEEauthorblockA{\textit{dept. name of organization (of Aff.)} \\
\textit{name of organization (of Aff.)}\\
City, Country \\
email address or ORCID}
\and
\IEEEauthorblockN{2\textsuperscript{nd} Given Name Surname}
\IEEEauthorblockA{\textit{dept. name of organization (of Aff.)} \\
\textit{name of organization (of Aff.)}\\
City, Country \\
email address or ORCID}
\and
\IEEEauthorblockN{3\textsuperscript{rd} Given Name Surname}
\IEEEauthorblockA{\textit{dept. name of organization (of Aff.)} \\
\textit{name of organization (of Aff.)}\\
City, Country \\
email address or ORCID}
\and

}

\maketitle

\begin{abstract}
This document is a model and instructions for \LaTeX.
This and the IEEEtran.cls file define the components of your paper [title, text, heads, etc.]. *CRITICAL: Do Not Use Symbols, Special Characters, Footnotes, 
or Math in Paper Title or Abstract.
\end{abstract}

\begin{IEEEkeywords}
component, formatting, style, styling, insert
\end{IEEEkeywords}
 攻撃モデル先に言いたい感ある.
\section{INTRODUCTION}
IP見ればすぐわかるから攻撃者はMITMパターンできますよっていう説明
\section{RELATED WORKS}
Rogue AP (RAP) detection methods are mainly classified into two categories: network administrator side detections and user side detections.
Network administrator side detections focus on the physical features such as Received Signal Strength Indication (RSSI) and clock skew which cannot be spoofed by an adversary.
RAP can be detected by comparing with the physical features of it with those in the predefined whitelist with equipments such as traffic sensors in each Wi-Fi network.

Wu et al. \cite{prapd} pay attention to the RSSI which is hard to be forged arbitrarily and highly correlated to the transmitter's location and power.
For each LAP in a network, RSSI, which is measured by additional costly devices, is registered as the information in whitelist beforehand.
By using RSSI, even if the MAC address of an AP is identical to that in the whitelist, that scheme can disclose that it is a RAP with spoofed MAC address set by an adversary at different location.
However, that scheme is hard to detect RAP which is located near the LAP because RSSI is not as exact as it can indicate a small difference of the nearby location. 
%Although it is useful as a supplementary feature, using only RSSI is insufficient to detect a RAP with high accuracy.

In order to detect in more detail, Lanze et al. \cite{clockskew} focus on clock skew as a device fingerprinting based purely on physical properties.
Clock skew is an unvoidable physical phenomenon that causes crystal oscillator based clocks to run with minuscule yet measurable deviations in speed.
However, these network administrator side detections are inapplicable to Wi-Fi hotspots because they have to setup additional sensors or install detection software in their infrastructure to prevent attacks besides providing free Internet service.
Thus, the detection schemes that require no equipment of additional devices by a network administrator are desired.

Meanwhile, user side detections do not need to introduce additional devices to a Wi-Fi hotspot. 
They focus on differences in the transmission characteristics caused by the extra hop to a RAP on the path between a LAP and user device. 
Compared with legitimate networks, extra hop results in several measurable changes in transmission characteristics such as Round Trip Time (RTT) and channel used between a user device and DNS server.

Mustafa et al. \cite{rtt} differentiate RAPs from LAPs by measuring the RTT between the user device and the DNS server through different target APs (RAPs or LAPs).
Because there exists the extra hop caused by the RAP on the path, RTT is longer in comparison to the case where a user directly connects to the LAP.
Although that scheme which leverages the packet delay are useful only for the case where the adversary sets RAP up on the laptop, Jang et al. \cite{previous} reveal the fact that the computational power of the software bridging mainly accounts for the packet delay.
Thus, the adversary can evade the packet delay based detection by utilizing hardware-based RAPs having little bridging delay unlike software-based RAPs.

In order to detect both types of RAPs, namely, software-based and hardware-based ones, \cite{previous} focuses on two communication channels utilized by a RAP between a user device and LAP.
Whereas a RAP intervene between a user device and LAP, two distinct channels are used to reduce communication delay caused by channel interference each other.
For example, it is assumed that channel 1 is used as the channel between a user device and a RAP, and channel 6 is that between a RAP and a LAP.
That scheme detects RAP by finding out the presence of these two channels with the throughput of the transmission from the user device to the DNS server.
That scheme is the most robust user side detection which is independent of the performance of the RAP because it is the countermeasure against a reasonable attack model where hardware-based RAP is used.
Thus, we select \cite{previous} as the previous scheme.
In the next section, we elaborate the previous scheme.

\section{ATTACK MODEL AND PREVIOUS SCHEME}
\subsection{Attack Model}
In an evil twin attack, the adversary sets up a RAP which uses a SSID of a LAP in the targeted Wi-Fi network.
Besides, the MAC address of the RAP is cloned from one of the LAPs in the network.
As a result, although a user device receives SSID broadcast from both LAP and the RAP, it cannot differentiate between these APs.
Thus, the user device simply connects with the AP that has a higher RSSI value.
We assume that a RAP relays WLAN traffic between a LAP and a user device, which act as a ``man-in-the-middle-attack'' to steal private information of a user.
By avoiding using mobile Internet access, e.g., 3G/4G, the adversary can evade simple detections with Internet Services Provider (ISP) names or Global IP addresses\cite{rtt}.
In addition to that, we assume that the adversary exploits hardware-based APs which cannot be detected accurately by existing schemes since they do not cause a computational delay due to a software bridging.

\subsection{Previous Scheme}
\subsubsection{Overview of the Previous Scheme}
The main idea of the previous scheme \cite{previous} is that the adversary needs to use two distinct communication channels on the path from a user device to the LAP to avoid channel interference each other.
The one is the channel for the path between a LAP and a RAP, and the other is that for the path between a RAP and the user device.
Thus, from the perspective of the user device, there exists another channel on the route that is different from the channel with the connected AP.
The extra channel cannot be observed directly from the user device.
The previous scheme detects the RAP by finding out these two channels on the basis of the decline in the throughput.
In order to decrease the throughput, the previous scheme saturates the channel used between a LAP and the RAP by intentionally interfering a channel with an additional equipment in a user device.
For example, when a user device is using channel 1 with the targeted AP which cannot be judged to be legitimacy, the equipment in a user device transmits a large number of packets to all the channel except channel 1 to saturate traffic on the path.
If there exists the other channel on the route, the decline in the throughput can be observed by the user device, and the presence of RAP is revealed.

\subsubsection{Shortcoming of the Previous Scheme}\label{sec:shortcoming}
Although the previous scheme is successful in the detection for hardware-based RAP in the experimental environment, it cannot detect accurately in the real world.
This is because throughput is considerably dependent on various factors of the network environment such as mobility of the traffic, collisions, network topology changes, and unintentional interference.
Since the traffic in real environment is unsteady, the previous scheme is subject to environmental changes, which can lead to degrade the accuracy.
Thus, the requirement that we must satisfy is to leverage factors which are independent of the network environment for the detection.

\section{PROPOSED SCHEME}
In order to meet the requirements mentioned in Section \ref{sec:shortcoming}, in this paper, we propose \title{}.
In the following subsections, we firstly explain the idea of the proposed scheme.
In the next subsection, the algorithm is described in detail.

\subsection{Idea}
The main idea of the proposed scheme is that there exist two APs, namely, RAP and LAP on the same path from a user device to a gateway.
In general, a LAP is the only device which exists on the same path from a user device to a gateway.
Therefore, in this case, the LAP is identical to the AP directly connected with a user device.
In an attack scenario, besides the connected RAP, a LAP exists inevitably on the other side of the RAP due to a man-in-the-middle-attack.
Therefore, a connected AP can be revealed as the RAP when the presence of a LAP is detected on the same path.

On the basis of this idea, the proposed scheme reveals that a LAP is on the other side of the connected RAP by finding out the MAC address of a LAP.
In order to discover the MAC address of a LAP on the same path, we leverage the phenomenon that a user device cannot receive Address Resolution Protocol (ARP) reply packets in the situation where there exist duplicate MAC addresses on the same path.
The proposed scheme intentionally creates such situation by setting the MAC address of a user device to the MAC addresses obtained from beacon frames of APs in the communication range of a user device.
Note that the MAC address of the AP with which a user device connects is excluded from targets for setting MAC addresses.
If the MAC address of a user device is set to that of a LAP on the same path, a LAP receives ARP reply packets whose original destination is a user device before a user device receives them.
Thus, since a user device cannot receive ARP reply packets, it continues to resend ARP requests, which results in disabling internet connectivity.
By observing the continuance of resending ARP request packets within a definite period of time without ARP reply packets, the proposed scheme can correctly reveal that there exists the RAP and a LAP on the path, which detects the attack.
When a user device connects with the RAP, there exists the MAC address of a LAP in the communication range of a user device.
This is because the RAP are located relatively near a LAP to avoid communication delay.
Hence, we can inevitably obtain the MAC address of a LAP in the case where there exists the RAP in a network.
In the real situation, it is possible that there exist several LAPs in a communication range of a user device.
Thus, we collect the only MAC addresses of APs which have the identical SSID to that of AP connected by a user device.
This is because RAP must utilize an SSID of a LAP for pretending to be LAP.
A user device can receive ARP reply packets even if its MAC address is set to that of each LAP on the distinct paths.
This is because the MAC addresses can be duplicated except those on the same path.
Since the proposed scheme is independent of the real network environment, it is useful for overcoming the shortcoming of the previous scheme.
In addition to that, the proposed scheme is not affected by a spoofed MAC address because it focuses on the only legitimate MAC address never spoofed.

\subsection{Algorithm}
In this subsection, the algorithm for detection based on searching the MAC address of LAP on the same path is explained.
The algorithm consists of three procedures, 1) AP information collection, 2) setting MAC address on user device, and 3) ARP observation.
The second and last procedures are repeatedly conducted for every MAC address in  $AP_{all}=\{A_i|0\le i \le n_{all}\}$, where $n_{all}$ is the number of all APs acquired in the first   phase.
Let $A_{target}$, $M_{A_{target}}$ and $S_{A_{target}}$ denote the targeted AP for detection which is directly connected, MAC address of $A_{target}$, and SSID of $A_{target}$, respectively.

\subsubsection{AP Information Collection}
In this procedure, the user device collects information of APs which exist in the communication range of a user device through beacon frames transmitted for establishment of communication.
Among these, APs whose SSID is same with that of the targeted AP, namely $S_{target}$, are extracted.
They are represented as $A_i$, whose MAC addresses are done as $M_{A_i}$.
At this stage, if there exists the AP which satisfy the formula: $M_{A_i} = M_{A_{target}}$  among $MAC_{all}=\{M_i|0 \le i \le n_{all}\}$, which means either of them clones another, it is detected as a RAP.
Thus, using the network should be avoided whether it is on the same path or not because the risk of being attacked is extremely high. 
In contrast, in the case where a RAP clones the MAC address which the LAP far from the user device has or it does not clone, we cannot detect only by checking the accordance of MAC addresses.
Thus the further investigation of the existence of another AP on the path is needed.

\subsubsection{Setting MAC Address on user device}
In this procedure, in order to reveal the existence of LAP which is connected on the other side of the targeted AP if  the user device is attacked, the situation is intentionally created where the MAC address of LAP is duplicated by the user device.
Thus, the MAC address of user device is set to $M_{Ai}$ in preparation for the next procedure.

\subsubsection{ARP Observation}
After the MAC address is set to $M_i$, the connection between the user device and $A_{target}$ is once lost since the gateway become unable to use original MAC address or $M_{i-1}$ as a destination for the packets.
Therefore, the gateway needs to upgrade the ARP table, in which the sets of MAC addresses and IP addresses of all the devices in the network are registered, so as to continue the communication with it.
Consequently, the ARP works automatically with new $M_i$ set to user device.
Thus, in this procedure, these ARP packets are observed so as to investigate whether the $A_i$ is on the same path with the user device. ここってso as toの主語とメインの主語が異なるけど問題ないのかなぁ.

%ARP is a procedure always done at the beginning of the Wi-Fi communication for mapping IP address to MAC address in a LAN to establish communication to the Internet and composed of two phases: ARP request, and ARP reply.
%ARP request is a packet sent by a user device to a gateway i.e. router of the LAN at the beginning of the network communication.
%Since MAC address is used for a destination address of a packet in LAN communication, the device cannnot communicate with the Internet, or even gateway, without the MAC address of the gateway.
%In order to acquire a MAC address of a gateway, ARP request packets, which includes source MAC address information, are sent to the gateway's IP address. 
%After the gateway get the packet from the device, ARP reply packets are sent by the gateway to the source device on the basis of the source MAC address included in ARP request packets, which can tell the gateway's MAC address by including it as a source MAC address.
As a result, in non attack scenario or an attack scenario of $A_i\ne LAP$, which means $A_i$ is on distinct path, ARP reply packets from the gateway are observed by the user device within the detection time, represented as $t$.
In contrast, in the case of $M_{A_i}=M_{LAP}$ in an attack scenario, ARP reply packets from the gateway cannot be observed since the LAP on the way back to the source of ARP request, i.e. the user device, receive them.
By searching such $M_{A_i}$ from $MAC_{all} = \{M_i|0\le i \le n_{all}\}$, the proposed scheme can reveal the $AP_{target}$ as a RAP.

\begin{thebibliography}{00}
\bibitem{snooping1} A. Adya, P. Bahl, R. Chandra, and L. Qiu, “Architecture and techniques for diagnosing faults in IEEE 802.11 infrastructure networks,”n Proc. of ACM Annual International Co.nference on Mobile Computing and Networking, MOBICOM, 2004, pp. 30-44.
\bibitem{snooping2} P. Bahl, R. Chandra, J. Padhye, L. Ravindranath, M. Singh, A. Wolman, and B. Zill, “Enhancing the security of corporate Wi-Fi etworks using DAIR,” in Proc. of ACM International Conference on Mobile Systems, Applications, and Services, MobiSys, 2006, pp. 1–14.
\bibitem{snooping3} R. Chandra, J. Padhye, A. Wolman, and B. Zill, “A location-based management system for enterprise wireless LANs,” in Proc. of USENIX Symposium on Networked Systems Design and Implementation NSDI, 2007.
\bibitem{rssi1} D. B. Faria and D. R. Cheriton, “Detecting identity-based attacks in wireless networks using signalprints,” in Proc. of ACM Workshop on Wireless Security, 2006, pp. 43–52.
\bibitem{rssi2} Y. Sheng, K. Tan, G. Chen, D. Kotz, and A. Campbell, “Detecting 802.11 MAC layer spoofing using received signal strength,” in Proc. of the 27th Conference on Computer Communications, INFOCOM, 2008.
\bibitem{clock1} F. Lanze, A. Panchenko, B. Braatz, and T. Engel, “Letting the puss in boots sweat: detecting fake access points using dependency of clock skews on temperature,” in Proc. of ACM Symposium on Information, Computer and Communications Security, ASIACCS, 2014, pp. 3–14.
\bibitem{clock2} S. Jana and S. K. Kasera, “On fast and accurate detection of unauthorized wireless access points using clock skews,” IEEE Trans. Mob. Comput., vol. 9, no. 3, pp. 449–462, 2010.
\bibitem{channel1} V. Brik, S. Banerjee, M. Gruteser, and S. Oh, “Wireless device identifica- tion with radiometric signatures,” in Proc. ofACM Annual International Conference on Mobile Computing and Networking, MOBICOM, 2008.
\bibitem{traceroute} S. Nikbakhsh, A. B. A. Manaf, M. Zamani, and M. Janbeglou, “A novel approach for rogue access point detection on the client-side,” 27th International Conference on Advanced Information Networking and Applications Workshops, 2012.
\bibitem{rtt} H. Han, B. Sheng, C. C. Tan, Q. Li, and S. Lu, “A timing-based scheme for rogue AP detection,” IEEE Trans. Parallel Distrib. Syst., vol. 22, no. 11, pp. 1912–1925, 2011
\bibitem{iat} C. Yang, Y. Song, and G. Gu, “Active user-side evil twin access point detection using statistical techniques,” IEEE Trans. Information Forensics and Security, vol. 7, no. 5, pp. 1638–1651, 2012.
\bibitem{previous} R. Jang, J. Kang, A. Mohaisen and D. Nyang, "Catch Me If You Can: Rogue Access Point Detection Using Intentional Channel Interference," in IEEE Transactions on Mobile Computing.

\end{thebibliography}
\vspace{12pt}
\end{document}
